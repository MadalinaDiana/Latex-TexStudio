\documentclass[a4paper,12pt]{amsbook}
\usepackage{graphicx} %png, jpg, pdf
\usepackage{graphics}
\usepackage{lipsum}
\usepackage[a4paper, twoside, includeheadfoot, left=0.75cm, hmarginratio=2cm, vmarginratio=0.17cm, bottom=0.25cm]{geometry}

\begin{document}
	\chapter{\Large Lansarea Android 1.0}
	\par	
   \begin{center}
   	În 2007, Apple a lansat primul iPhone și a introdus o nouă eră în domeniul calculelor mobile. La acea vreme, Google încă lucra în secret pe Android, dar în noiembrie a acelui an, compania a început încet să-și dezvăluie planurile de a concura cu Apple și alte platforme mobile. Într-o dezvoltare majoră, Google a condus formarea a ceea ce se numea Open Handset Alliance. Acesta a inclus producători de telefoane precum HTC și Motorola, producători de cipuri precum Qualcomm și Texas Instruments și operatori, inclusiv T-Mobile.\\
   \end{center}
	\vspace{0.1cm}
	\begin{center}
	Apoi, președintele și CEO-ul Google, Eric Schmidt, a spus citat: „Anunțul de astăzi este mai ambițios decât orice„ telefon Google ”despre care presa a speculat în ultimele câteva săptămâni. Viziunea noastră este că platforma puternică pe care o prezentăm va alimenta mii de modele de telefoane diferite. ”//
	\par
	\end{center}
\begin{figure}[h]
	\centering
	\includegraphics[width=0.7\linewidth]{android1.0}
	\caption[android]{android1.0}
	\label{fig:android1}
\end{figure}
\par
\chapter{\Large Lansarea Android 1.5 Briose}
\par	
Primul nume de cod public oficial pentru Android nu a apărut până când versiunea 1.5 Briosa a fost lansată în aprilie 2009. Creditul pentru denumirea versiunilor Android după bomboane dulci și deserturi a revenit în mod tradițional managerului său de proiect de la Google, Ryan Gibson. Cu toate acestea, motivele sale specifice pentru utilizarea unei astfel de convenții de numire rămân necunoscute.

Briosa  a adăugat câteva caracteristici noi și îmbunătățiri comparativ cu primele două versiuni publice. Aceasta include lucruri pe care le luăm acum de la sine înțeles, cum ar fi posibilitatea de a încărca videoclipuri pe YouTube, o modalitate prin care ecranele telefoanelor se rotesc automat și suport pentru tastaturi de la terți.

Unele dintre telefoanele lansate cu Briosa
\begin{figure}[h]
	\centering
	\includegraphics[width=0.7\linewidth]{Briosa}
	\caption[Briosa]{Andriod 1.5 Briosa}
	\label{fig:briosa}
\end{figure}
 instalat din cutie au inclus primul telefon Samsung Galaxy, împreună cu HTC Hero.
 \par
 \chapter{\Large Lansarea Android 1.6 Gogoasa}
 \par
Google a lansat rapid Android 1.6 Gogoasa în septembrie 2009. Noul sistem de operare oferă acum suport pentru operatorii care utilizează rețele bazate pe CDMA. Acest lucru a permis telefoanelor Android să fie vândute de toți operatorii din întreaga lume.

Alte caracteristici au inclus introducerea casetei de căutare rapidă și comutarea rapidă între cameră, cameră video și galerie pentru a eficientiza experiența de captare media. Donut a introdus, de asemenea, widget-ul Power Control pentru gestionarea Wi-Fi, Bluetooth, GPS etc.
\par
Unul dintre telefoanele vândute cu Gogoasa instalat a fost nefastul Dell Streak. Avea un ecran imens (la acea vreme) de 5 inci și a fost descris pe site-ul nostru drept „smartphone / tabletă”. În prezent, un ecran de 5 inci este considerat relativ mic pentru un smartphone.	
\par
\begin{figure}[h]
	\centering
	\includegraphics[width=0.7\linewidth]{Gogoasa}
	\caption[Gogoasa]{Android 1.4 Gogoasa}
	\label{fig:gogoasa}
\end{figure}
 \par
\chapter{\Large Lansarea Android 2.0-2.1 Ecler }
\par

În octombrie 2009 - aproximativ un an după lansarea Android 1.0 - Google a lansat versiunea 2.0 a sistemului de operare, cu numele de cod oficial Eclair. Această versiune a fost prima care a adăugat suport text-to-speech și a introdus, de asemenea, imagini de fundal live, suport pentru mai multe conturi și navigare pe Google Maps, printre multe alte caracteristici noi și îmbunătățiri.
\\
Motorola Droid a fost primul telefon care a purtat Android 2.0 din cutie. Droid a fost, de asemenea, primul telefon bazat pe Android care a fost vândut de Verizon Wireless. Într-o bucată amuzantă de trivia, în timp ce Google a folosit în siguranță Android ca nume pentru sistemul său de operare, termenul „Droid” a fost înregistrat de Lucasfilm, referindu-se la roboții francizei Star Wars. Motorola a trebuit să obțină permisiunea și să plătească niște bani lui Lucasfilm pentru a folosi numele pentru telefonul său. Motorola a continuat să utilizeze marca Droid pentru multe dintre telefoanele sale până în 2016.\\
\par

	\begin{figure}[h]
		\centering
		\includegraphics[width=0.7\linewidth]{Ecler}
		\caption[Ecler]{Android 2.0-2.1 Ecler}
		\label{fig:ecler}
	\end{figure}
 \par
\chapter{\Large Lansarea Android 2.2 Iaurt Inghetat }
\par
Android 2.2 Froyo (prescurtare pentru „iaurt înghețat”) a fost lansat oficial în mai 2010. Smartphone-urile sportive Froyo ar putea profita de mai multe funcții noi, inclusiv funcții de hotspot mobil Wi-Fi, notificări push prin intermediul serviciului Android Cloud to Device Messaging (C2DM) , suport pentru bliț și multe altele.

Primul smartphone care a purtat brandul Google Nexus - Nexus One - a fost lansat cu Android 2.1 din cutie la începutul anului 2010, dar a primit rapid o actualizare over-the-air la Froyo mai târziu în acel an. Aceasta a marcat o nouă abordare pentru Google, compania lucrând mai aproape ca oricând cu producătorul de hardware HTC pentru a prezenta Android pur.

\begin{figure}[h]
	\centering
	\includegraphics[width=0.7\linewidth]{"Iaurt inghtat"}
	\caption[Iaurt inghetat]{Andriod 2.2 Iaurt inghetat}
	\label{fig:iaurt-inghtat}
\end{figure}	
\par
	\chapter{\Large Lansarea Android 2.3 Turta dulce}
	\par
	Android 2.3 Gingerbread a fost lansat în septembrie 2010. Sistemul de operare a primit o reîmprospătare a interfeței de utilizator sub Gingerbread. Acesta a adăugat suport pentru utilizarea funcțiilor de comunicare în câmp aproape (NFC) pentru smartphone-uri cu hardware-ul necesar. Primul telefon care a purtat atât Gingerbread, cât și hardware NFC a fost Nexus S, care a fost co-dezvoltat de Google și Samsung. Gingerbread a pus, de asemenea, bazele pentru selfie, adăugând suport pentru mai multe camere și asistență pentru chat video în cadrul Google Talk.
	\begin{figure}[h]
		\centering
		\includegraphics[width=0.7\linewidth]{"Turta dulce"}
		\caption[Turta dulce]{Andriod 2.3 Turta dulce}
		\label{fig:turta-dulce}
	\end{figure}
\par
\chapter{\Large Lansarea Android 3.0 Fagure}
\par	
//
Această versiune a sistemului de operare este probabil ciudat. Honeycomb a fost creat pentru tablete și alte dispozitive mobile cu afișaje mai mari. A fost introdus pentru prima dată în februarie 2011, împreună cu tableta Motorola Xoom. Acesta a inclus funcții precum o interfață de utilizare reproiectată pentru ecrane mari, împreună cu o bară de notificare plasată în partea de jos a ecranului unei tablete.
//
Ideea a fost că Honeycomb ar oferi funcții care nu ar putea fi gestionate de afișajele mai mici găsite pe smartphone-uri la acea vreme. A fost, de asemenea, un răspuns al Google și al partenerilor săi terți la lansarea în 2010 a iPad-ului Apple. Chiar dacă Honeycomb era disponibil, unele tablete au fost încă lansate cu versiunile Android 2.x bazate pe smartphone. În cele din urmă, Honeycomb a ajuns să fie o versiune de Android care nu a fost adoptată pe scară largă. Google a decis să integreze majoritatea funcțiilor sale în următoarea sa versiune majoră 4.0, Ice Cream Sandwich. Este un pic anormal în istoria Android.//
\begin{figure}[h]
	\centering
	\includegraphics[width=0.7\linewidth]{Faugure}
	\caption[Fagure]{Android 4.0 Fagure}
	\label{fig:faugure}
\end{figure}

\par
\chapter{\Large Lansarea Android 4.0 Înghețată de tip sandwich}
\par
Lansată în octombrie 2011, versiunea Ice Cream Sandwich de Android a adus o serie de funcții noi. A combinat multe dintre opțiunile versiunii Honeycomb numai pentru tablete cu turta dulce orientată spre smartphone. De asemenea, a inclus o „tavă de favorite” pe ecranul de pornire, împreună cu primul suport pentru deblocarea unui telefon prin utilizarea camerei sale pentru a face o fotografie a feței proprietarului. Acest tip de asistență de conectare biometrică a evoluat și s-a îmbunătățit considerabil de atunci.

Alte modificări notabile cu ICS au inclus suport pentru toate butoanele de pe ecran, gesturi de glisare pentru a respinge notificările și filele browserului și posibilitatea de a vă monitoriza utilizarea datelor prin mobil și Wi-Fi.
\begin{figure}[h]
	\centering
	\includegraphics[width=0.7\linewidth]{"Inghetata de tip sandwich"}
	\caption[Inghetata de tip sandwich]{Android 4.0 Inghetata de tip sandwich}
	\label{fig:inghetata-de-tip-sandwich}
\end{figure}
\par
\chapter{\Large Lansarea Android 4.1-4.3 Jeleuri}
\par
Era Jelly Bean a Android a început în iunie 2012 odată cu lansarea Android 4.1. Google a lansat rapid versiunile 4.2 și 4.3 - ambele sub eticheta Jelly Bean - în octombrie 2012 și, respectiv, în iulie 2013.
Unele dintre noile adăugiri din aceste actualizări de software au inclus noi funcții de notificare care afișau mai multe butoane de conținut sau acțiune, împreună cu suport complet pentru versiunea Android a browserului web Chrome Google, care a fost inclusă în Android 4.2. Google Now și-a făcut apariția ca parte a Căutării, în timp ce „Project Butter” a fost introdus pentru a accelera animațiile și pentru a îmbunătăți capacitatea de reacție Android la atingere. Afișajele externe și Miracast au câștigat, de asemenea, suport, la fel ca și fotografia HDR.
\begin{figure}[h]
	\centering
	\includegraphics[width=0.7\linewidth]{Jeleuri}
	\caption[Jeleuri]{Android 4.1-4.3 Jeleuri}
	\label{fig:jeleuri}
\end{figure}
\par
\chapter{\Large Lansarea Android 4.4 KitKat}
\par
Android 4.4 este prima versiune a sistemului de operare care a folosit de fapt un nume înregistrat anterior pentru o bomboană. Înainte de lansarea oficială în septembrie 2013, compania a lansat indicii la conferința Google I / O din acel an că numele de cod pentru Android 4.4 ar fi de fapt „Key Lime Pie”. Într-adevăr, majoritatea echipei Android de la Google credea că va fi și cazul.

După cum sa dovedit, directorul parteneriatelor globale Android pentru Google, John Lagerling, a crezut că „Key Lime Pie” nu ar fi un nume suficient de familiar pentru a fi folosit în întreaga lume. În schimb, a decis să facă ceva diferit. El a contactat Nestle, creatorii barului KitKat și i-a întrebat dacă pot folosi numele pentru Android 4.4. Nestle a fost de acord și KitKat a devenit numele următoarei versiuni Android. A fost un experiment în marketing pe care Google nu l-a reaprins până la lansarea Oreo (vom ajunge la asta).

KitKat nu avea un număr mare de funcții noi, dar avea un lucru care a contribuit într-adevăr la extinderea pieței globale a Android. A fost optimizat pentru a rula pe smartphone-uri care aveau 512 MB RAM. Acest lucru le-a permis producătorilor de telefoane să utilizeze cea mai recentă versiune de Android pe telefoane mult mai ieftine. Smartphone-ul Google Nexus 5 a fost primul cu Android 4.4 preinstalat.
\begin{figure}[h]
	\centering
	\includegraphics[width=0.7\linewidth]{Kitkat}
	\caption[KitKat]{Android 4.4 KitKat}
	\label{fig:kitkat}
\end{figure}
\par
\chapter{\Large Lansarea Android 5.0 Acadea}
\par
Lansat pentru prima dată în toamna anului 2014, Android 5.0 Lollipop a fost un shakeup major în aspectul general al sistemului de operare. A fost prima versiune a sistemului de operare care a folosit noul limbaj Google Design material. A făcut o utilizare liberală a efectelor de iluminare și de umbră, printre altele, pentru a simula un aspect asemănător hârtiei pentru interfața de utilizare Android. UI a primit, de asemenea, alte câteva actualizări, inclusiv o bară de navigare renovată, notificări bogate pentru ecranul de blocare și multe altele.

Actualizarea ulterioară a Android 5.1 a făcut câteva alte modificări sub capotă. Aceasta a inclus asistență oficială pentru apeluri dual-SIM, HD Voice și Protecție dispozitiv pentru a păstra hoții blocați de pe telefon chiar și după o resetare din fabrică.

Smartphone-ul Google Nexus 6, împreună cu tableta Nexus 9, au fost primele dispozitive care au preinstalat Lollipop.
\begin{figure}[h]
	\centering
	\includegraphics[width=0.7\linewidth]{Acadea}
	\caption[Acadea]{Android 5.0 Acadea}
	\label{fig:acadea}
\end{figure}
\par
\chapter{\Large Lansarea Android 6.0 Bezea}
\par
Lansat în toamna anului 2015, Android 6.0 Marshmallow a folosit dulceața preferată de rulote ca simbol principal. Pe plan intern, Google a folosit „Macadamia Nut Cookie” pentru Android 6.0 înainte de anunțul oficial Marshmallow. Acesta a inclus funcții precum un nou sertar cu aplicații cu defilare verticală, împreună cu Google Now on Tap, suport nativ pentru deblocarea biometrică a amprentelor digitale, suport USB de tip C, introducerea Android Pay (acum Google Pay) și multe altele.

Primele dispozitive livrate cu Marshmallow preinstalate au fost smartphone-urile Google Nexus 6P și Nexus 5X, împreună cu tableta Pixel C.
\begin{figure}[h]
	\centering
	\includegraphics[width=0.7\linewidth]{Bezea}
	\caption[Bezea]{Android 6.0 Bezea}
	\label{fig:bezea}
\end{figure}
\par
\chapter{\Large Lansarea Android 7.0 Nuga}
\par
Versiunea 7.0 a sistemului de operare mobil Google a fost lansată în toamna anului 2016. Înainte ca Nougat-ul să fie dezvăluit, „Android N” a fost denumit intern de Google drept „New York Cheesecake”. Numeroasele caracteristici noi ale lui Nougat au inclus funcții multitasking mai bune pentru numărul tot mai mare de smartphone-uri cu afișaje mai mari, cum ar fi modul cu ecran divizat, împreună cu trecerea rapidă între aplicații.

Google a făcut și o serie de mari schimbări în culise. A trecut la un nou compilator JIT pentru a accelera aplicațiile, a acceptat API-ul Vulkan pentru o redare 3D mai rapidă și a permis OEM-urilor să-și susțină acum platforma Daydream VR dispărută.

Google a folosit, de asemenea, versiunea pentru a face un impuls îndrăzneț pe piața smartphone-urilor premium. Pixel și Pixel XL ale companiei, împreună cu LG V20, au fost primele lansate cu Nougat preinstalat.
\begin{figure}[h]
	\centering
	\includegraphics[width=0.7\linewidth]{Nuga}
	\caption[Nuga]{Android 7.0 Nuga}
	\label{fig:nuga}
\end{figure}
\par
\chapter{\Large Lansarea Android 8.0 Oreo}
\par
În martie 2017, Google a anunțat și a lansat oficial prima previzualizare pentru dezvoltatori pentru Android O, cunoscut și ca Android 8.0. Înainte de această lansare, Hiroshi Lockheimer, vicepreședintele senior al Android la Google, a postat un GIF al unui tort Oreo pe Twitter - primul indiciu solid că Oreo, popularul cookie, ar fi într-adevăr numele de cod oficial pentru Android 8.0.

În august, Google a confirmat numele public inspirat de cookie-uri pentru Android 8.0. A fost a doua oară în care compania a ales un nume comercial pentru Android (Oreo este deținut de Nabisco). Într-o pauză de la tradiția sa, Google a arătat statuia mascotei Android Oreo pentru prima dată într-un eveniment de presă din New York, mai degrabă decât la sediul său Googleplex. Statuia descrie mascota Android ca un supererou zburător, completat cu o pelerină. O a doua statuie a fost pusă la sediul principal al Google mai târziu în acea zi.

În ceea ce privește funcțiile, Android Oreo include multe modificări vizuale în meniul Setări, împreună cu suport nativ pentru modul imagine în imagine, canale de notificare, noi API-uri de completare automată pentru o mai bună gestionare a parolelor și a datelor de completare și multe altele. Android Oreo a venit prima dată instalat pe propriile telefoane Google Pixel 2.
\begin{figure}[h]
	\centering
	\includegraphics[width=0.7\linewidth]{Oreo}
	\caption[Oreo]{Android 8.0 Oreo}
	\label{fig:oreo}
\end{figure}
\par
\chapter{\Large Lansarea Android 9.0 Plăcintă}
\par
Google a lansat prima previzualizare pentru dezvoltatori a următoarei actualizări majore Android, Android 9.0 P, pe 7 martie 2018. Pe 6 august 2018, compania a lansat oficial versiunea finală a Android 9.0, oferindu-i numele de cod oficial „Pie”.
\par
Android 9.0 Pie a inclus o serie de noi caracteristici majore și modificări. Unul dintre ei a renunțat la butoanele de navigare tradiționale în favoarea unui buton alungit din centru, care a devenit noul buton de acasă. Glisând în sus de acesta apare Prezentare generală, cu cele mai recente aplicații utilizate, o bară de căutare și cinci sugestii de aplicații în partea de jos. Puteți glisa la stânga pentru a vedea toate aplicațiile deschise recent sau puteți glisa butonul de pornire spre dreapta pentru a derula rapid aplicațiile dvs.
\par
Android 9.0 Pie a inclus, de asemenea, câteva funcții noi concepute pentru a ajuta la prelungirea duratei de viață a bateriei smartphone-ului. Acest lucru a fost realizat prin utilizarea învățării automate pe dispozitiv, care prezice ce aplicații veți utiliza acum și ce aplicații nu veți utiliza până mai târziu. Pie are și Shush, o caracteristică care vă pune automat telefonul în modul Nu deranjați atunci când răsuciți ecranul telefonului pe o suprafață plană. Există, de asemenea, Slices, care oferă o versiune mai mică a unei aplicații instalate în Căutarea Google, oferind anumite funcții ale aplicației fără a deschide aplicația completă.
\par
Ca de obicei, Android 9.0 Pie a fost disponibil mai întâi oficial pentru telefoanele Google Pixel, dar a fost lansat și pe telefonul esențial în același timp.
\begin{figure}[h]
	\centering
	\includegraphics[width=0.7\linewidth]{Placinta}
	\caption[Placinta]{Android 9.0 Placinta}
	\label{fig:placinta}
\end{figure}
\par
\chapter{\Large Lansarea Android 10}
\par
La 10 ani de la lansarea sistemului de operare, am obținut o altă etapă importantă a istoriei Android. Google a lansat prima previzualizare oficială a dezvoltatorului Android Q, pe 13 martie 2019. Pe 22 august 2019, Google a anunțat o actualizare majoră a mărcii Android. Aceasta a inclus un nou logo și, mai important, decizia de a renunța la numele tradițional de desert pentru următoarea versiune. Drept urmare, Android Q este cunoscut oficial la fel de Android 10. A fost lansat oficial pe 3 septembrie 2019, pentru dispozitivele Google Pixel.
\par
Ca de obicei cu orice versiune nouă de Android, Android 10 a avut o serie de funcții și îmbunătățiri noi, precum și o serie de API-uri noi. Aceasta a inclus asistență pentru graba telefoanelor pliabile viitoare de atunci. Android 10 a introdus, de asemenea, un mod întunecat la nivelul întregului sistem, împreună cu noi comenzi de navigare prin gesturi, un meniu de partajare mai eficient, funcții de răspuns inteligent pentru toate aplicațiile de mesagerie și un control mai mare asupra permisiunilor bazate pe aplicații
\begin{figure}[h]
	\centering
	\includegraphics[width=0.7\linewidth]{"Android 10"}
	\caption[Android 10]{Android 10}
	\label{fig:android-10}
\end{figure}
\par
\chapter{\Large Lansarea Android 11}
\par
Android 11 a sosit cu multe funcții noi. Aceasta include o nouă categorie de notificări Conversații în care toate chat-urile dvs. din diferite aplicații sunt colectate într-un singur loc. De asemenea, aveți opțiunea de a salva fiecare notificare care a apărut pe telefonul dvs. în ultimele 24 de ore. O funcție complet nouă vă permite să înregistrați ecranul telefonului, completat cu sunet, fără a avea nevoie de o aplicație terță parte. Există, de asemenea, o nouă secțiune din Android 11 dedicată controlului dispozitivelor inteligente de acasă.
Cu toate acestea, telefoanele Pixel primesc o caracteristică exclusivă pentru Android 11. Utilizează AI și învățarea automată pentru a controla ce aplicații apar pe docul telefonului.
\par
Google și-a montat statuia tradițională pentru a celebra lansarea Android 11, dar a lansat și o versiune AR a statuii pentru toate telefoanele Android ARCore. Are chiar și câteva ouă de Paște, inclusiv o rețetă pentru prepararea tortului de catifea roșie. Acesta se întâmplă, de asemenea, să fie numele de cod intern pentru sistemul de operare de la Google.
\begin{figure}[h]
	\centering
	\includegraphics[width=0.7\linewidth]{"Android 11"}
	\caption[Android 11]{Android 11}
	\label{fig:android-11}
\end{figure}

\end{document}