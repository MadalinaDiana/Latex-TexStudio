\documentclass[a4paper]{article}
\usepackage[left=2cm, right=2cm, top=17mm]{geometry}
\usepackage{graphicx}
\usepackage{lipsum}
\usepackage{xcolor}
\graphicspath{ {../inserare img/./imagini/} }
\begin{document}
\vspace{5mm}
	\begin{center}
\large\textbf{Lansarea Android OS 1.0}
\end{center}

\begin{figure}[h]
	\centering
	\includegraphics[width=10cm,height=10cm,keepaspectratio]{Android01} \\
	\caption{Versiunea Android OS 1.0}
	\label{fig:Android01}
\end{figure}
\par
În 2007, Apple a lansat primul iPhone și a inaugurat o nouă eră în domeniul calculelor mobile. La acea vreme, Google încă lucra în secret pe Android, dar în noiembrie a aceluiași an, compania a început încet să-și dezvăluie planurile de a concura cu Apple și alte platforme mobile. Într-o dezvoltare majoră, Google a condus formarea a ceea ce se numea Open Handset Alliance. Acesta a inclus producători de telefoane precum HTC și Motorola, producători de cipuri precum Qualcomm și Texas Instruments și operatori, inclusiv T-Mobile.
\par

\begin{center}
	\large\textbf{Lansarea Android 1.5 Cupcake}
\end{center}

\begin{figure}[h]
	\centering
	\includegraphics[width=10cm,height=10cm,keepaspectratio]{Android15} \\
	\caption{Versiunea Android 1.5 Cupcake}
	\label{fig:Android15}
\end{figure}
\par
Primul nume de cod public oficial pentru Android nu a apărut până când versiunea 1.5 Cupcake a fost lansată în aprilie 2009. Creditul pentru denumirea versiunilor Android după bomboane dulci și deserturi a revenit în mod tradițional managerului său de proiect de la Google, Ryan Gibson. Cu toate acestea, motivele sale specifice pentru utilizarea unei astfel de convenții de numire rămân necunoscute.
\clearpage
\begin{center}
	\large\textbf{Lansarea Android 1.6 Donut}
\end{center}

\begin{figure}[h]
	\centering
	\includegraphics[width=10cm,height=10cm,keepaspectratio]{Android16} \\
	\caption{Versiunea Android 1.6 Donut}
	\label{fig:Android16}
\end{figure}
\par Google a lansat rapid Android 1.6 Donut în septembrie 2009. Noul sistem de operare oferă acum suport pentru operatorii care utilizează rețele bazate pe CDMA. Acest lucru a permis telefoanelor Android să fie vândute de către toți operatorii din întreaga lume.
\begin{center}
	\large\textbf{Lansarea Android 2.0 Eclair}
\end{center}

\begin{figure}[h]
	\centering
	\includegraphics[width=10cm,height=10cm,keepaspectratio]{Android20} \\
	\caption{Versiunea Android 2.0 Eclair}
	\label{fig:Android20}
\end{figure}
\par
În octombrie 2009 - la aproximativ un an după lansarea Android 1.0 - Google a lansat versiunea 2.0 a sistemului de operare, cu numele de cod oficial Eclair. Această versiune a fost prima care a adăugat suport text-to-speech și a introdus, de asemenea, imagini de fundal vii, suport pentru mai multe conturi și navigare pe Google Maps, printre multe alte caracteristici noi și îmbunătățiri.
\clearpage
\begin{center}
	\large\textbf{Lansarea Android 2.2 Froyo}
\end{center}

\begin{figure}[h]
	\centering
	\includegraphics[width=10cm,height=10cm,keepaspectratio]{Android22} \\
	\caption{Versiunea Android 2.2 Froyo}
	\label{fig:Android22}
\end{figure}
\par
Android 2.2 Froyo (prescurtare pentru iaurt înghețat) a fost lansat oficial în mai 2010. Smartphone-urile sportive Froyo ar putea profita de mai multe funcții noi, inclusiv funcții de hotspot mobil Wi-Fi, notificări push prin intermediul serviciului Android Cloud to Device Messaging (C2DM) , suport pentru bliț și multe altele.
\begin{center}
	\large\textbf{Lansarea Android 2.3 Gingerbread}
\end{center}

\begin{figure}[h]
	\centering
	\includegraphics[width=10cm,height=10cm,keepaspectratio]{Android23} \\
	\caption{Versiunea Android 2.3 Gingerbread}
	\label{fig:Android23}
\end{figure}
\par
Android 2.3 Gingerbread a fost lansat în septembrie 2010. Sistemul de operare a primit o reîmprospătare a interfeței de utilizator sub Gingerbread. Acesta a adăugat suport pentru utilizarea funcțiilor de comunicare în câmp aproape (NFC) pentru smartphone-uri cu hardware-ul necesar. Primul telefon care a purtat atât Gingerbread, cât și hardware NFC a fost Nexus S, care a fost co-dezvoltat de Google și Samsung. Gingerbread a pus, de asemenea, bazele pentru selfie, adăugând suport pentru mai multe camere și asistență pentru chat video în cadrul Google Talk.
\clearpage
\begin{center}
	\large\textbf{Lansarea Android 3.0 Honeycomb}
\end{center}

\begin{figure}[h]
	\centering
	\includegraphics[width=10cm,height=10cm,keepaspectratio]{Android30} \\
	\caption{Versiunea Android 3.0 Honeycomb}
	\label{fig:Android30}
\end{figure}
\par
Această versiune a sistemului de operare este probabil ciudat. Honeycomb a fost creat pentru tablete și alte dispozitive mobile cu afișaje mai mari. A fost introdus pentru prima dată în februarie 2011, împreună cu tableta Motorola Xoom. Acesta a inclus funcții precum o interfață de utilizare reproiectată pentru ecrane mari, împreună cu o bară de notificare plasată în partea de jos a ecranului unei tablete.
\begin{center}
	\large\textbf{Lansarea Android 4.0 Ice Cream Sandwich}
\end{center}

\begin{figure}[h]
	\centering
	\includegraphics[width=10cm,height=10cm,keepaspectratio]{Android40} \\
	\caption{Versiunea Android 4.0 Ice Cream Sandwich}
	\label{fig:Android40}
\end{figure}
\par
Lansată în octombrie 2011, versiunea Ice Cream Sandwich de Android a adus o serie de funcții noi. A combinat multe dintre opțiunile versiunii Honeycomb numai pentru tablete cu Gingerbread orientat spre smartphone. De asemenea, a inclus o tavă de favorite pe ecranul de pornire, împreună cu primul suport pentru deblocarea unui telefon folosind camera foto pentru a face o fotografie a feței proprietarului. Acest tip de asistență de conectare biometrică a evoluat și s-a îmbunătățit considerabil de atunci.
.
\clearpage
\begin{center}
	\large\textbf{Lansarea Android 4.1-4.3 Jelly Bean}
\end{center}

\begin{figure}[h]
	\centering
	\includegraphics[width=10cm,height=10cm,keepaspectratio]{Android41} \\
	\caption{Versiunea Android 4.1-4.3 Jelly Bean}
	\label{fig:Android41}
\end{figure}
\par
Era Jelly Bean a Android a început în iunie 2012 odată cu lansarea Android 4.1. Google a lansat rapid versiunile 4.2 și 4.3 - ambele sub eticheta Jelly Bean - în octombrie 2012 și, respectiv, în iulie 2013.
\par
Unele dintre noile adăugiri din aceste actualizări de software au inclus noi funcții de notificare care afișau mai multe butoane de conținut sau acțiune, împreună cu suport complet pentru versiunea Android a browserului web Chrome Google, care a fost inclusă în Android 4.2. Google Now și-a făcut apariția ca parte a Căutării, în timp ce Project Butter a fost introdus pentru a accelera animațiile și pentru a îmbunătăți capacitatea de reacție Android. Afișajele externe și Miracast au câștigat, de asemenea, suport, la fel ca și fotografiile HDR.
\begin{center}
	\large\textbf{Lansarea Android 4.4 KitKat}
\end{center}

\begin{figure}[h]
	\centering
	\includegraphics[width=10cm,height=10cm,keepaspectratio]{Android44} \\
	\caption{Versiunea Android 4.4 KitKat}
	\label{fig:Android44}
\end{figure}
\par
Android 4.4 este prima versiune a sistemului de operare care a folosit de fapt un nume înregistrat anterior pentru o bucată de bomboane. Înainte de lansarea oficială în septembrie 2013, compania a lansat în acel an sugestii la conferința Google I / O că numele de cod pentru Android 4.4 ar fi de fapt \textit{Key Lime Pie}. Într-adevăr, majoritatea echipei Android de la Google a crezut că va fi și cazul.
\clearpage
\begin{center}
	\large\textbf{Lansarea Android 5.0 Lollipop}
\end{center}

\begin{figure}[h]
	\centering
	\includegraphics[width=10cm,height=10cm,keepaspectratio]{Android50} \\
	\caption{Versiunea Android 5.0 Lollipop}
	\label{fig:Android50}
\end{figure}
\par
Lansat pentru prima dată în toamna anului 2014, Android 5.0 Lollipop a fost un shakeup major în aspectul general al sistemului de operare. A fost prima versiune a sistemului de operare care a folosit noul limbaj Google Design material. A făcut o utilizare liberală a efectelor de iluminare și de umbră, printre altele, pentru a simula un aspect asemănător hârtiei pentru interfața cu utilizatorul Android. UI a primit, de asemenea, alte câteva actualizări, inclusiv o bară de navigare renovată, notificări bogate pentru ecranul de blocare și multe altele.
\begin{center}
	\large\textbf{Lansarea Android 6.0 Marshmallow}
\end{center}

\begin{figure}[h]
	\centering
	\includegraphics[width=10cm,height=10cm,keepaspectratio]{Android60} \\
	\caption{Versiunea Android 6.0 Marshmallow}
	\label{fig:Android60}
\end{figure}
\par
Lansat în toamna anului 2015, Android 6.0 Marshmallow a folosit dulceața preferată de rulote ca simbol principal. Pe plan intern, Google a folosit Macadamia Nut Cookie pentru Android 6.0 înainte de anunțul oficial Marshmallow. Acesta a inclus funcții precum un nou sertar cu aplicații cu defilare verticală, împreună cu Google Now on Tap, suport nativ pentru deblocarea biometrică a amprentelor digitale, suport USB de tip C, introducerea Android Pay (acum Google Pay ) și multe altele.
\clearpage
\begin{center}
	\large\textbf{Lansarea Android 7.0 Nougat}
\end{center}

\begin{figure}[h]
	\centering
	\includegraphics[width=10cm,height=10cm,keepaspectratio]{Android70} \\
	\caption{Versiunea Android 7.0 Nougat}
	\label{fig:Android70}
\end{figure}
\par
Versiunea 7.0 a sistemului de operare mobil Google a fost lansată în toamna anului 2016. Înainte ca Nougat-ul să fie dezvăluit, \textit{Android N} a fost denumit intern de Google drept New York Cheesecake. Numeroasele caracteristici noi ale nougatului includeau funcții multi-tasking mai bune pentru numărul tot mai mare de smartphone-uri cu afișaje mai mari, cum ar fi modul cu ecran divizat, împreună cu comutarea rapidă între aplicații.
\begin{center}
	\large\textbf{Lansarea Android 8.0 Oreo}
\end{center}
\begin{figure}[h]
	\centering
	\includegraphics[width=10cm,height=10cm,keepaspectratio]{Android80} \\
	\caption{Versiunea Android 8.0 Oreo}
	\label{fig:Android80}
\end{figure}
\par
În martie 2017, Google a anunțat și a lansat oficial prima previzualizare pentru dezvoltatori pentru Android O, cunoscut și ca Android 8.0. Înainte de această lansare, Hiroshi Lockheimer, vicepreședintele senior al Android la Google, a postat un GIF al unui tort Oreo pe Twitter - primul indiciu solid că Oreo, popularul cookie, ar fi într-adevăr numele de cod oficial pentru Android 8.0.
\clearpage
\begin{center}
	\large\textbf{Lansarea Android 9.0 Pie}
\end{center}

\begin{figure}[h]
	\centering
	\includegraphics[width=10cm,height=10cm,keepaspectratio]{Android90} \\
	\caption{Versiunea Android 9.0 Pie}
	\label{fig:Android90}
\end{figure}
\par
Google a lansat prima previzualizare pentru dezvoltatori a următoarei actualizări majore Android, Android 9.0 P, pe 7 martie 2018. Pe 6 august 2018, compania a lansat oficial versiunea finală a Android 9.0, oferindu-i numele de cod oficial Pie.
\par
Android 9.0 Pie a inclus o serie de noi caracteristici majore și modificări. Unul dintre ei a renunțat la butoanele de navigare tradiționale în favoarea unui buton alungit din centru, care a devenit noul buton de acasă.
\begin{center}
	\large\textbf{Lansarea Android 10}
\end{center}

\begin{figure}[h]
	\centering
	\includegraphics[width=10cm,height=10cm,keepaspectratio]{Android10} \\
	\caption{Versiunea Android 10}
	\label{fig:Android10}
\end{figure}
\par
La 10 ani de la lansarea sistemului de operare, am obținut o altă etapă importantă a istoriei Android. Google a lansat prima previzualizare oficială a dezvoltatorului Android Q, pe 13 martie 2019. Pe 22 august 2019, Google a anunțat o actualizare majoră a mărcii Android . Aceasta a inclus un nou logo și, mai important, decizia de a renunța la numele tradițional de desert pentru următoarea versiune. Drept urmare, Android Q este cunoscut oficial la fel de Android 10. A fost lansat oficial pe 3 septembrie 2019 pentru dispozitivele Google Pixel.
\clearpage
\begin{center}
	\large\textbf{Lansarea Android 11}
\end{center}

\begin{figure}[h]
	\centering
	\includegraphics[width=10cm,height=10cm,keepaspectratio]{Android11} \\
	\caption{Versiunea Android 11}
	\label{fig:Android11}
\end{figure}
\par
Pe 18 februarie, Google a lansat prima Developer Preview pentru Android 11. După ce au fost lansate mai multe versiuni beta publice, versiunea finală a Android 11 a fost lansată pe 8 septembrie 2020.
\par
Android 11 a sosit cu multe funcții noi. Aceasta include o nouă categorie de notificări Conversații în care toate chat-urile dvs. din diferite aplicații sunt colectate într-un singur loc. De asemenea, aveți opțiunea de a salva fiecare notificare care a apărut pe telefonul dvs. în ultimele 24 de ore. O funcție nouă vă permite să înregistrați ecranul telefonului, completat cu sunet, fără a avea nevoie de o aplicație terță parte. Există, de asemenea, o nouă secțiune din Android 11 dedicată controlului dispozitivelor inteligente de acasă.
\end{document}